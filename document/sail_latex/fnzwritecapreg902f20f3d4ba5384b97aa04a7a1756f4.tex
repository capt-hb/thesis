function #\hyperref[zwriteCapReg]{writeCapReg}#(n, cap) =
  if (n == 0b00000) then
    ()
  else {
    let i = #\hyperref[zunsigned]{unsigned}#(n);
    if trace then {
      #\hyperref[zprerr]{prerr}#(#\hyperref[zstringzyofzyint]{string\_of\_int}#(i));
      #\hyperref[zprerr]{prerr}#(" <- ");
      #\hyperref[zprerrzyendline]{prerr\_endline}#(#\hyperref[zcapToString]{capToString}#(cap, false));
      /* Additionally check that the cap we are storing is in normal
         form i.e. it is unchanged by round-tripping through bits.
         This is quite a strong check because caps might differ from
         normal form in ways that don't really #\hyperref[zmatter]{matter}# (e.g. otype
         non-zero for unsealed capability) but it is probably a good
         idea to maintain this invariant. It's disabled if not tracing
         because it is slow. We might be able to eliminate the
         non-normal values with a better type... */
      let cap2 = #\hyperref[zcapBitsToCapability]{capBitsToCapability}#(cap.tag, #\hyperref[zcapToBits]{capToBits}#(cap));
      if (cap != cap2) then {
        #\hyperref[zprerrzyendline]{prerr\_endline}#("Wrote non-normal cap:");
        #\hyperref[zprerrzyendline]{prerr\_endline}#(#\hyperref[zcapToString]{capToString}#(cap, false));
        #\hyperref[zprerrzyendline]{prerr\_endline}#(#\hyperref[zcapToString]{capToString}#(cap2, false));
        assert(false, "wrote non-normal capability");
      };
    } else {
      #\hyperref[zskipzyescape]{skip\_escape}#();
    };
    (*CapRegs[i]) = cap;
  }
