function clause #\hyperref[zexecute]{execute}# (#\hyperref[zCClearTags]{CClearTags}#(cb)) =
{
  #\hyperref[zcheckCP2usable]{checkCP2usable}#();
  let cb_val = #\hyperref[zreadCapRegDDC]{readCapRegDDC}#(cb);
  if not (cb_val.tag) then
    #\hyperref[zraisezyc2zyexception]{raise\_c2\_exception}#(CapEx_TagViolation, cb)
  else if cb_val.sealed then
    #\hyperref[zraisezyc2zyexception]{raise\_c2\_exception}#(CapEx_SealViolation, cb)
  else if not (cb_val.permit_store) then
    #\hyperref[zraisezyc2zyexception]{raise\_c2\_exception}#(CapEx_PermitStoreViolation, cb)
  else
  {
    let vAddr   = #\hyperref[zgetCapCursor]{getCapCursor}#(cb_val);
    let vAddr64 = #\hyperref[ztozybits]{to\_bits}#(64, #\hyperref[zgetCapCursor]{getCapCursor}#(cb_val));
    if (vAddr + caps_per_cacheline * cap_size) > #\hyperref[zgetCapTop]{getCapTop}#(cb_val) then
      #\hyperref[zraisezyc2zyexception]{raise\_c2\_exception}#(CapEx_LengthViolation, cb)
    else if vAddr < #\hyperref[zgetCapBase]{getCapBase}#(cb_val) then
      #\hyperref[zraisezyc2zyexception]{raise\_c2\_exception}#(CapEx_LengthViolation, cb)
    else if not(vAddr % (cap_size * caps_per_cacheline) == 0) then
      #\hyperref[zSignalExceptionBadAddr]{SignalExceptionBadAddr}#(AdEL, vAddr64)
    else
    {
      let pAddr  = #\hyperref[zTLBTranslate]{TLBTranslate}#(vAddr64, StoreData);
      foreach(i from 0 #\hyperref[zto]{to}# (caps_per_cacheline - 1))
      {
        /* We could use write_tag_bool instead of reading data and writing it out again
           but that would be incompatible with the way proofs are currently done. 
           There are concurrency implications to this method. */
        let (_, mem) = #\hyperref[zMEMrzytagged]{MEMr\_tagged}#(pAddr + i*cap_size, cap_size, false);
        #\hyperref[zMEMwzytagged]{MEMw\_tagged}#(pAddr + i*cap_size, cap_size, false, mem);
      }
    }
  }
}
