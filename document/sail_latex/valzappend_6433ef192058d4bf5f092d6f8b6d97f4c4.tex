append_64 : forall ('n : Int). (bits('n), bits(64)) -> bits('n + 64)